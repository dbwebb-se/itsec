\documentclass{article}
\usepackage[utf8]{inputenc}
\usepackage[swedish]{babel}

\usepackage{biblatex}
\addbibresource{references.bib}

\usepackage{hyperref}
\hypersetup{
    colorlinks=true,
    linkcolor=blue,
    filecolor=magenta,
    urlcolor=blue,
}

\title{Utbildningens paketering - en möjlighet för att nå breddad rekrytering och en jämnare könsfördelning i tekniska utbildningar}
\author{Mikael Roos (mos@bth.se)}
\date{Augusti 2020}

\begin{document}

\maketitle

\section*{Abstrakt}

BLa bla.


\section*{Nyckelord}

BLa bla.


\section*{Om författaren}

BLa bla.


\section*{Inledning}

Som programansvarig till tekniska och programmeringsinriktade utbildningar har jag ibland fått önskemål från skolans ledning om att försöka attrahera fler kvinnliga sökanden till utbildningen. Främst gäller denna önskan till kandidatprogrammet "Webbprogrammering 180hp" (web180) som ges på campus där 1 av 10 är kvinnliga sökanden. En tanke är att en blandad studentgrupp erbjuder en önskvärd studiemiljö och i detta specifika fallet är kvinnliga sökanden i klar minoritet.

Skolverkets statistik för 2019/2020 \cite{skolverket_2020} rörande könsfördelning på gymnasiets högskoleförberedande Teknikprogram påtalar att endast 2 av 10 tjejer väljer den tekniska banan. Det är endast Teknikprogrammet som har en övervikt av killar. Naturprogrammet och Ekonomiprogrammet har en jämn könsfördelning och programmen Humanistiska och Samhällsvetenskap har en tydlig övervikt av tjejer. Möjligen är könsfördelningen i de högskoleförberedande gymnasieprogrammen ett tecken på att man kan förvänta sig en liknande könsfördelningen i de tekniska högskoleprogrammen.

Skolverket har utrett könsskillnader i utbildning och inom högskolan är det Teknik-sektorn som har allra minst närvaro av kvinnliga studenter, även om andelen kvinnliga studenter verkar öka med tiden. Tittar man andelen kvinnor i inom olika yrken så är det inom Ingenjör/Civilingenjör som det finns minst kvinnor. När man tittar på åldersfördelningen så är det en större andel yngre kvinnor som är aktiva i branschen. Det går sakta men över tiden verkar det som vi går mot en högre andel kvinnor i traditionell tekniska yrken. \cite{skolverket_287}

\begin{quote}
"De traditionella val som kvinnor och män gör i gymnasieskolan, fortsätter i den högre utbildningen. Samtidigt visar statistiken att könssammansättningen inom högre utbildning förändrats genom att kvinnor gör inbrytningar på traditionellt manliga utbildningar. Andelen kvinnor som utexaminerats inom teknikområdet har fördubblats under de senaste tjugofem åren." \cite{skolverket_287}
\end{quote}

Av Sveriges Ingenjörers medlemmar var 28 procent av civilingenjörerna och högskoleingenjörerna kvinnor år 2018. Detta enligt en rapport från Kungliga Ingenjörsvetenskapsakademin.

\begin{quote}
"Ett område, teknikområdet, är dock fortfarande tydligt dominerat av män och av de studerande vid svenska ingenjörsutbildningar (högskoleingenjör och civilingenjör) utgör kvinnorna endast cirka 30 procent av de studerande. Det är intressant att notera att ingenjörsutbildningarna och officersutbildningen nu är de enda yrkesutbildningarna inom högskolan där män är i majoritet och där könsbalansen är utanför 60 procent – 40 procent kriteriet." \cite{iva_2019}
\end{quote}

Enligt rapporten stipulerar "40 procent kriteriet" att en situation är jämställd om inget kön utgör mer än 60\% eller mindre än
40\%.

För att återkoppla dessa siffror till söktrycket på campusprogrammet web180 med ett förhållande 1 till 10 så har vi en situation som enligt IVA inte kan anses som jämställd. Men hur kan situationen förändras, bortsett från att se tiden an?



\section*{Bakgrund}

BLa bla.


\section*{Observation och hypotes}

BLa bla.


\section*{Metod}

BLa bla.


\section*{Resultat}

BLa bla.


\section*{Analys}

BLa bla.


\section*{Slutsats}

BLa bla.


\section*{Framtida arbete}

BLa bla.


\printbibliography

\end{document}
